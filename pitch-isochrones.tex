\documentclass[table]{beamer}
%%\documentclass[handout]{beamer}

\mode<presentation>
{
  \definecolor{navitialight}{RGB}{126,186,200}
  \definecolor{navitiadark}{RGB}{76,102,114}

  \useoutertheme[subsection=false]{miniframes}
  %%\useoutertheme[footline=authortitle]{miniframes}
  \usecolortheme[named=navitiadark]{structure}
  %%\usecolortheme{dolphin}
  \usecolortheme{orchid}
  \useinnertheme{circles}
  \setbeamerfont{block title}{size=\normalsize}
  \setbeamercovered{transparent}

  %%% le foot pour avoir la numérotation des slides %%%
  \setbeamertemplate{footline}{%
    \leavevmode%
    \hbox{%
      \begin{beamercolorbox}[wd=.5\paperwidth,ht=2.5ex,dp=1.125ex,
        leftskip=.3cm plus1fill,rightskip=.3cm]{author in head/foot}%
        \usebeamerfont{title in head/foot}\insertshorttitle
      \end{beamercolorbox}%
      \begin{beamercolorbox}[wd=.5\paperwidth,ht=2.5ex,dp=1.125ex,
        leftskip=.3cm,rightskip=.3cm plus1fil]{title in head/foot}%
        \usebeamerfont{author in head/foot}\insertshortauthor\hfill
        \insertframenumber/\inserttotalframenumber
      \end{beamercolorbox}%
    }%
    \vskip0pt%
  }

  \setbeamercolor{palette primary}{fg=white,bg=navitiadark}
  \setbeamercolor{palette secondary}{fg=white,bg=navitialight}
  \setbeamercolor{palette tertiary}{fg=white,bg=navitiadark}
  \setbeamercolor{palette quaternary}{fg=white,bg=navitialight}
}

\mode<handout>
{
  \usepackage{pgfpages}
  \pgfpagesuselayout{4 on 1}[a4paper,border shrink=5mm,landscape]
}

\usepackage[utf8]{inputenc}
\usepackage{lmodern}
\usepackage[T1]{fontenc}
\usepackage[english,francais]{babel}
\usepackage{multirow}
\usepackage{hhline}

\newcommand{\nologo}{\setbeamertemplate{logo}{}}

\newenvironment{foreignpar}[1][english]{%
    \em\selectlanguage{#1}%
}{}
\newcommand*{\foreign}[2][english]{%
    \emph{\foreignlanguage{#1}{#2}}%
}

\title{Pitch isochrones}

\author{Zoé Brunet}

\institute[Kisio Digital] % (optional, but mostly needed)
{
  Kisio Digital\\
  20 rue Hector Malot\\
  75012 Paris, France}
%% - Use the \inst command only if there are several affiliations.
%% - Keep it simple, no one is interested in your street address.

\date{25/08/16}
%% - Either use conference name or its abbreviation.
%% - Not really informative to the audience, more for people (including
%%   yourself) who are reading the slides online

%% If you have a file called "university-logo-filename.xxx", where xxx
%% is a graphic format that can be processed by latex or pdflatex,
%% resp., then you can add a logo as follows:
\pgfdeclareimage[width=.2\linewidth]{logo}{images/logo_nio}
\logo{\pgfuseimage{logo}\hspace{.04\linewidth}}


%% Delete this, if you do not want the table of contents to pop up at
%% the beginning of each subsection:
\AtBeginSection[]
{
  \begin{frame}<beamer>
    \frametitle{Table des matières}
    \tableofcontents[currentsection,hideothersubsections]
  \end{frame}
}
\AtBeginSubsection[]
{
  \begin{frame}<beamer>
    \frametitle{Table des matières}
    \tableofcontents[currentsection,subsectionstyle=show/shaded/hide]
  \end{frame}
}


\begin{document}

\begin{frame}
  \titlepage
\end{frame}

\section{Définition}

\begin{frame}
  \frametitle{Présentation}

  \begin{description}
  \item[Isochrone:] ensemble des points atteignables en un temps donné à partir d'un point de départ.
  \end{description}
\end{frame}

\begin{frame}
  \frametitle{À quoi ça peut ressembler ?}
  \centering
  \includegraphics[width=0.7\linewidth]<1>{images/ancien_isochrone}
  \includegraphics[width=0.7\linewidth]<2>{images/iso_stif}
  \vfill
\end{frame}

\begin{frame}
  \frametitle{A quoi ça sert ?}
    \begin{description}
    \item[Filtrage]
    \begin{itemize} 
      \item Trouver les magasins de BD à proximité de chez moi
      \item Application Bisous 
    \end{itemize}
  \end{description}
   \begin{description}
    \item[Cartographie]
      \begin{itemize} 
        \item Agence immobilière
        \item Ville
      \end{itemize}
  \end{description}
\end{frame}

\section{Filtrage}

\begin{frame}
  \frametitle{Démo !}
  \centering
  \includegraphics[width=0.9\linewidth]<1>{images/step_1}
  \includegraphics[width=0.9\linewidth]<2>{images/step_2}
  \includegraphics[width=0.9\linewidth]<3>{images/step_3}
  \includegraphics[width=0.9\linewidth]<4>{images/step_4}
  \includegraphics[width=0.9\linewidth]<5>{images/step_7}
  \includegraphics[width=0.9\linewidth]<6>{images/step_8}
  \vfill
\end{frame}

\section{Cartographie}

\begin{frame}
  \frametitle{L'API isochrones}
  \centering
  \includegraphics[width=0.9\linewidth]<1>{images/iso_classique}
  \includegraphics[width=0.9\linewidth]<2>{images/iso_min}
  \includegraphics[width=0.8\linewidth]<3>{images/multi_iso}
  \includegraphics[width=0.7\linewidth]<4>{images/iso_sncf}
  \vfill
\end{frame}

\begin{frame}
    \frametitle{Cartographie}
    \centering
    \includegraphics[width=0.8\linewidth]<1>{images/isochrone_Boston}
\end{frame}
  
\begin{frame}
  \frametitle{Surprise !}
  \begin{description}
  \item[Heat map:] Nouvelle API qui utilise le filaire de voierie
  \end{description}

  \begin{columns}
    \begin{column}{0.5\linewidth}
    \centering
    \includegraphics[width=\linewidth]<1>{images/comp_iso_fin}
    \end{column}
    \begin{column}{0.5\linewidth}
    \centering
    \includegraphics[width=\linewidth]<1>{images/comp_raster_fin}
    \end{column}    
  \end{columns}
\end{frame}
 
\begin{frame}
  \frametitle{Comparaison}

  \centering
  \begin{tabular} {|l|c|c|}
  \hline
  Critère & Isochrone & Heat map \\
  \hline
  Temps de réponse & < 1s &  < 3 s \\
  \hline
  Taille du flux & < 1 Mo & 4 Mo \\
  \hline
  Facilement utilisable pour faire des filtres & ++ & + \\
  \hline
   Précision des temps d'acces & + & ++ \\
  \hline
  Prend en compte la voierie à l'arrivée & -- & ++ \\
  \hline
  facilement intégrable & ++ & + \\
  \hline
  \end{tabular}

\end{frame}

\section{Conclusion}

\begin{frame}
  \frametitle{Conclusion}
  \begin{description}
  \item[]
    \begin{itemize} 
        \item En prod !
        \item Sur playground
        \newline
        \item Pas dans NMP
        \item pas dans le plug and play
        \newline
        \item heat maps à finaliser
    \end{itemize}
  \end{description}
\end{frame}

\begin{frame}
  \titlepage
\end{frame}

\end{document}

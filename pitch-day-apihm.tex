\documentclass[table]{beamer}
%%\documentclass[handout]{beamer}

\mode<presentation>
{
  \definecolor{ctpred}{RGB}{233,38,73}
  \definecolor{ctporange}{RGB}{243,97,60}

  \useoutertheme[subsection=false]{miniframes}
  %%\useoutertheme[footline=authortitle]{miniframes}
  \usecolortheme[named=ctpred]{structure}
  %%\usecolortheme{dolphin}
  \usecolortheme{orchid}
  \useinnertheme{circles}
  \setbeamerfont{block title}{size=\normalsize}
  \setbeamercovered{transparent}

  %%% le foot pour avoir la numérotation des slides %%%
  \setbeamertemplate{footline}{%
    \leavevmode%
    \hbox{%
      \begin{beamercolorbox}[wd=.5\paperwidth,ht=2.5ex,dp=1.125ex,
        leftskip=.3cm plus1fill,rightskip=.3cm]{author in head/foot}%
        \usebeamerfont{title in head/foot}\insertshorttitle
      \end{beamercolorbox}%
      \begin{beamercolorbox}[wd=.5\paperwidth,ht=2.5ex,dp=1.125ex,
        leftskip=.3cm,rightskip=.3cm plus1fil]{title in head/foot}%
        \usebeamerfont{author in head/foot}\insertshortauthor\hfill
        \insertframenumber/\inserttotalframenumber
      \end{beamercolorbox}%
    }%
    \vskip0pt%
  }

  \setbeamercolor{palette primary}{fg=white,bg=ctporange}
  \setbeamercolor{palette secondary}{fg=white,bg=ctpred}
  \setbeamercolor{palette tertiary}{fg=white,bg=ctporange}
  \setbeamercolor{palette quaternary}{fg=white,bg=ctpred}
}

\mode<handout>
{
  \usepackage{pgfpages}
  \pgfpagesuselayout{4 on 1}[a4paper,border shrink=5mm,landscape]
}

\usepackage[utf8]{inputenc}
\usepackage{lmodern}
\usepackage[T1]{fontenc}
\usepackage[english,francais]{babel}
\usepackage{multirow}
\usepackage{hhline}

\newcommand{\nologo}{\setbeamertemplate{logo}{}}

\newenvironment{foreignpar}[1][english]{%
    \em\selectlanguage{#1}%
}{}
\newcommand*{\foreign}[2][english]{%
    \emph{\foreignlanguage{#1}{#2}}%
}

\title{Pitch Day APiHM}

\author{Happy Team Friends}

\institute[Kisio Digital] % (optional, but mostly needed)
{
  Kisio Digital\\
  20 rue Hector Malot\\
  75012 Paris, France}
%% - Use the \inst command only if there are several affiliations.
%% - Keep it simple, no one is interested in your street address.

\date{Pitch day SWAT 2}
%% - Either use conference name or its abbreviation.
%% - Not really informative to the audience, more for people (including
%%   yourself) who are reading the slides online

%% If you have a file called "university-logo-filename.xxx", where xxx
%% is a graphic format that can be processed by latex or pdflatex,
%% resp., then you can add a logo as follows:
%\pgfdeclareimage[width=.2\linewidth]{logo}{images/canaltp}
%\logo{\pgfuseimage{logo}\hspace{.04\linewidth}}


%% Delete this, if you do not want the table of contents to pop up at
%% the beginning of each subsection:
\AtBeginSection[]
{
  \begin{frame}<beamer>
    \frametitle{Table des matières}
    \tableofcontents[currentsection,hideothersubsections]
  \end{frame}
}
\AtBeginSubsection[]
{
  \begin{frame}<beamer>
    \frametitle{Table des matières}
    \tableofcontents[currentsection,subsectionstyle=show/shaded/hide]
  \end{frame}
}


\begin{document}

\begin{frame}
  \titlepage    
\end{frame}

\begin{frame}%%[allowframebreaks]
  \frametitle{Table des matières}
  \tableofcontents[hideallsubsections]
  %% You might wish to add the option [pausesections]
\end{frame}

\section{Introduction}

\begin{frame}
  \frametitle{Introduction}

  API navitia

  Besoin d'interface, plusieurs essaies, pas de solution satisfaisante

  Une équipe de choc pour résoudre ce problème
\end{frame}

\section{L'idée}

\begin{frame}[fragile]
  \frametitle{Utiliser l'API}
  
  \centering\includegraphics[width=\linewidth]{images/curl}
\end{frame}

\begin{frame}
  \frametitle{Quelques chiffres navitia.io}

  \begin{itemize}
  \item Sur 1360 inscrits à navitia.io, 100 ont réalisés au moins
    1~appel le mois dernier (environ 7\%).
  \item Time to market des réutilisateurs estimé entre 3~à 10~mois.
    Cette donnée devra être validée et approfondie à l'occasion d'une
    étude qualitative auprès des réutilisateurs.
  \end{itemize}
\end{frame}

\begin{frame}
  \frametitle{Direction}

  \begin{description}
  \item[Problématique] Comment réduire les délais de création et de
    mise sur le marché des produits/services utilisant l'API
    navitia.io ?
  \item[Objectifs business] Augmenter les usages de l'API.
  \item[Buts de la solution] navitia.io companion :
    \begin{itemize}
    \item Aider à interroger l'API.
    \item Comprendre la réponse.
    \item Comprendre la philosophie.
    \item Aider à prototyper $\Rightarrow$ favoriser l'imagination et
      l'inspiration des réutilisateurs
    \end{itemize}
  \end{description}
\end{frame}

\begin{frame}
  \frametitle{À quoi ça peut ressembler ?}

  TODO: Simulation maquette pour montrer les idées «~question~»,
  «~réponse~», «~navigation~».
\end{frame}

\section{Environnement}

\begin{frame}
  \frametitle{Solutions concurrentes interne/externe}

\end{frame}

\begin{frame}
  \frametitle{Solutions concurrentes interne/externe}
  \begin{description}
    \item[Interne: Demo.Navitia.io]
  \end{description}
  \centering\includegraphics[width=\linewidth]{images/demo_navitia_io}
\end{frame}

\begin{frame}
  \frametitle{Solutions concurrentes interne/externe}
  \begin{description}
    \item[Interne: Navitia Explorer]
  \end{description}
  \centering\includegraphics[width=\linewidth]{images/navitia_explorer}
\end{frame}

\begin{frame}
  \frametitle{Solutions concurrentes interne/externe}
  \begin{description}
    \item[Interne: IHM Artemis]
  \end{description}
  \centering\includegraphics[width=\linewidth]{images/ihm-artemis}
\end{frame}

\begin{frame}
  \frametitle{Solutions concurrentes interne/externe}
  \begin{description}
    \item[Externe: Swagger]
  \end{description}
  \centering\includegraphics[width=\linewidth]{images/swagger}
\end{frame}

\begin{frame}
  \frametitle{Contexte marché dynamique}
  \begin{description}
    \item[Acteurs historiques : ]
  \end{description}
  \begin{itemize}
    \item Google: 
    \centering\includegraphics[width=0.5\textwidth]{images/google_transit}
    \item Nokia : Here 
    \centering\includegraphics[width=0.5\textwidth]{images/nokia_here}
  \end{itemize}
\end{frame}

\begin{frame}
  \frametitle{Contexte marché dynamique}
  \begin{description}
    \item[Acteurs transit centric : ]
  \end{description}
  \begin{itemize}
    \item Citymapper : API et widgets 
    \item Waynaut  :  API et Widget
    \item Transport Api : API seulement
  \end{itemize}
\end{frame}

\begin{frame}
  \frametitle{Contexte marché dynamique}
  \begin{description}
    \item[Acteurs open source : ]
  \end{description}
  \begin{itemize}
    \item OpenTripPlanner/RRRRR
  \end{itemize}
\end{frame}

\begin{frame}
  \frametitle{Contexte marché dynamique}
  \begin{description}
    \item[Points Forts : ]
  \end{description}
  \begin{itemize}
    \item ???
  \end{itemize}
  \begin{description}
    \item[Points Manquants : ]
  \end{description}
  \begin{itemize}
    \item ???
  \end{itemize}
\end{frame}


\begin{frame}
  \frametitle{Audience}

  TODO
\end{frame}

\section{Les pépettes}

\begin{frame}
  \frametitle{Outillage pour la force de vente}

  TODO: citation de Dorian
\end{frame}

\begin{frame}
  \frametitle{Gains de temps en interne sur l'utilisation de navitia
    grâce à APiHM}

  \centering

  \begin{tabular}{|c|c|c|c|c|}
    \hline
    \multirow{2}{*}{Équipe}& \multicolumn{2}{c|}{par
      semaine}&nombre de&\multirow{2}{*}{Gain\footnote{1j = 7h,
        1 an = 40 semaines, 202j/an = 1 ETP}}\\
    \hhline{~--~~}
    & Avant & Après & personnes &\\
    \hline
    Data       &15h &10h & 6 & 171j/an\\
    Support    & 3h & 1h & 6 &  69j/an\\
    RO         & 2h & 1h & 8 &  46j/an\\
    NMP Mobile & 3h &1h30& 4 &  34j/an\\
    NMP Web    & 1h &40min&6 &  11j/an\\
    Q\&A       & 2h &2h30& 2 &   6j/an\\
    \hline
    \multicolumn{4}{|r|}{Total} &
    337j/an = 1{,}7 ETP\\
    \hline
  \end{tabular}
\end{frame}

\begin{frame}
  \frametitle{Avantage concurrentiel}

  
\end{frame}

\begin{frame}
  \frametitle{Business model}

  TODO
\end{frame}

\section{Organisation}

\begin{frame}
  \frametitle{Roadmap}

  TODO
\end{frame}

\begin{frame}
  \frametitle{L'après SWAT}

  TODO
\end{frame}

\section{Conclusion}

\begin{frame}
  \frametitle{Conclusion}

  TODO
\end{frame}

\end{document}

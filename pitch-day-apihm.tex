\documentclass[table]{beamer}
%%\documentclass[handout]{beamer}

\mode<presentation>
{
  \definecolor{navitialight}{RGB}{126,186,200}
  \definecolor{navitiadark}{RGB}{76,102,114}

  \useoutertheme[subsection=false]{miniframes}
  %%\useoutertheme[footline=authortitle]{miniframes}
  \usecolortheme[named=navitiadark]{structure}
  %%\usecolortheme{dolphin}
  \usecolortheme{orchid}
  \useinnertheme{circles}
  \setbeamerfont{block title}{size=\normalsize}
  \setbeamercovered{transparent}

  %%% le foot pour avoir la numérotation des slides %%%
  \setbeamertemplate{footline}{%
    \leavevmode%
    \hbox{%
      \begin{beamercolorbox}[wd=.5\paperwidth,ht=2.5ex,dp=1.125ex,
        leftskip=.3cm plus1fill,rightskip=.3cm]{author in head/foot}%
        \usebeamerfont{title in head/foot}\insertshorttitle
      \end{beamercolorbox}%
      \begin{beamercolorbox}[wd=.5\paperwidth,ht=2.5ex,dp=1.125ex,
        leftskip=.3cm,rightskip=.3cm plus1fil]{title in head/foot}%
        \usebeamerfont{author in head/foot}\insertshortauthor\hfill
        \insertframenumber/\inserttotalframenumber
      \end{beamercolorbox}%
    }%
    \vskip0pt%
  }

  \setbeamercolor{palette primary}{fg=white,bg=navitiadark}
  \setbeamercolor{palette secondary}{fg=white,bg=navitialight}
  \setbeamercolor{palette tertiary}{fg=white,bg=navitiadark}
  \setbeamercolor{palette quaternary}{fg=white,bg=navitialight}
}

\mode<handout>
{
  \usepackage{pgfpages}
  \pgfpagesuselayout{4 on 1}[a4paper,border shrink=5mm,landscape]
}

\usepackage[utf8]{inputenc}
\usepackage{lmodern}
\usepackage[T1]{fontenc}
\usepackage[english,francais]{babel}
\usepackage{multirow}
\usepackage{hhline}

\newcommand{\nologo}{\setbeamertemplate{logo}{}}

\newenvironment{foreignpar}[1][english]{%
    \em\selectlanguage{#1}%
}{}
\newcommand*{\foreign}[2][english]{%
    \emph{\foreignlanguage{#1}{#2}}%
}

\title{Pitch Day APiHM}

\author{Happy Team Friends}

\institute[Kisio Digital] % (optional, but mostly needed)
{
  Kisio Digital\\
  20 rue Hector Malot\\
  75012 Paris, France}
%% - Use the \inst command only if there are several affiliations.
%% - Keep it simple, no one is interested in your street address.

\date{Pitch day SWAT 2}
%% - Either use conference name or its abbreviation.
%% - Not really informative to the audience, more for people (including
%%   yourself) who are reading the slides online

%% If you have a file called "university-logo-filename.xxx", where xxx
%% is a graphic format that can be processed by latex or pdflatex,
%% resp., then you can add a logo as follows:
\pgfdeclareimage[width=.2\linewidth]{logo}{images/logo_nio}
\logo{\pgfuseimage{logo}\hspace{.04\linewidth}}


%% Delete this, if you do not want the table of contents to pop up at
%% the beginning of each subsection:
\AtBeginSection[]
{
  \begin{frame}<beamer>
    \frametitle{Table des matières}
    \tableofcontents[currentsection,hideothersubsections]
  \end{frame}
}
\AtBeginSubsection[]
{
  \begin{frame}<beamer>
    \frametitle{Table des matières}
    \tableofcontents[currentsection,subsectionstyle=show/shaded/hide]
  \end{frame}
}


\begin{document}

\begin{frame}
  \titlepage
\end{frame}

\section{L'idée}

\begin{frame}
  \frametitle{Présentation}

  \begin{description}
  \item[navitia] Une suite logiciel exposant une API REST HATEOAS d'information voyageur.
  \item[navitia.io] Un service ouvert proposant l'API navitia sur des
    données opendata dans le monde entier hébergé par Kisio Digital.
  \end{description}
\end{frame}

\begin{frame}
  \frametitle{Audience}

  Les utilisateurs de navitia:
  \begin{itemize}
  \item Intégrateurs externes (développeurs, startup, SSII et pure
    players web).
  \item En interne:
    \begin{itemize}
    \item Équipes techniques (intégrateurs internes, data, RO);
    \item Support;
    \item Commercial.
    \end{itemize}
  \end{itemize}
\end{frame}

\begin{frame}
  \frametitle{Utiliser l'API}
  
  \centering
  \includegraphics[width=\linewidth]<1>{images/curl}
  \includegraphics[width=0.5\linewidth]<2>{images/firefox}
\end{frame}

\begin{frame}
  \frametitle{Direction}

  \begin{description}
  \item[Problématique] Comment réduire les délais de création et de
    mise sur le marché des produits/services utilisant l'API navitia?
  \item[Objectifs business] Augmenter les usages de l'API, Développer
    la confiance par l'adoption.
  \item[Buts de la solution] navitia studio:
    \begin{itemize}
    \item Aider à interroger l'API.
    \item Comprendre la réponse.
    \item Comprendre la philosophie.
    \item Aider à prototyper $\Rightarrow$ favoriser l'imagination et
      l'inspiration des réutilisateurs
    \end{itemize}
  \end{description}
\end{frame}

\begin{frame}
  \frametitle{À quoi ça peut ressembler ?}

  \centering
  \includegraphics[width=0.7\linewidth]<1>{images/step_1}
  \includegraphics[width=0.7\linewidth]<2>{images/step_2}
  \includegraphics[width=0.7\linewidth]<3>{images/step_3}
  \includegraphics[width=0.7\linewidth]<4>{images/step_4}
  \includegraphics[width=0.7\linewidth]<5>{images/step_5}
  \includegraphics[width=0.7\linewidth]<6>{images/step_6}
  \vfill
\end{frame}

\section{Environnement}

\begin{frame}
  \frametitle{Solutions concurrentes internes/externes}
  \begin{description}
    \item[Interne]
      \begin{itemize} 
        \item Demo.Navitia.io
        \item Navitia Explorer
        \item IHM Artemis
      \end{itemize}
  \end{description}
    \begin{description}
    \item[Externe] 
    \begin{itemize} 
      \item Swagger 
      \item MeShape 
      \item Plugin Json Firefox
      \item Json Viewer
      \item ...
    \end{itemize}
  \end{description}
\end{frame}

\begin{frame}
  \frametitle{Contexte marché dynamique}
  \begin{description}
  \item[Concurrents qui ont déjà des solutions similaires: ]\strut\par
    \begin{itemize}
    \item Nokia Here Transit: RESTFUL/JS API Explorer
    \item Transport Api : API seulement, UK seulement
    \item Waynaut :  API et Widgets (MeShape)
    \item OpenTripPlanner: Interface Debug
    \end{itemize}
  \item[Concurrents qui n'ont pas de solutions similaires: ]\strut\par
    \begin{itemize}
    \item Citymapper : API et Widgets 
    \item Google Transit: Documentation
    \end{itemize}
  \end{description}
\end{frame}

\section{Proposition de valeur}

\begin{frame}
  \frametitle{Améliorer la productivité des collaborateurs dans
    l'intégration et le débogage de navitia}

  Aide à la formation des nouveaux arrivants.

  Gains de temps en interne sur l'utilisation de navitia
  grâce à APiHM
  \vfill
  \centering
  \begin{tabular}{|c|c|c|c|c|}
    \hline
    \multirow{2}{*}{Équipe}& \multicolumn{2}{c|}{par
      semaine}&nombre de&\multirow{2}{*}{Gain\footnote{1j = 7h,
        1 an = 40 semaines, 200j/an = 1 ETP}}\\
    \hhline{~--~~}
    & Avant & Après & personnes &\\
    \hline
    Data       &15h &10h & 6 & 171j/an\\
    Support    & 3h & 1h & 6 &  69j/an\\
    RO         & 2h & 1h & 8 &  46j/an\\
    NMP Mobile & 3h &1h30& 4 &  34j/an\\
    NMP Web    & 1h &40min&6 &  11j/an\\
    Q\&A       & 2h &1h30& 2 &   6j/an\\
    \hline
    Total      &26h &15h40&32&337j/an = 1{,}7 ETP\\
    \hline
  \end{tabular}
\end{frame}

\begin{frame}
  \frametitle{Outillage pour la force de vente}

  La principale difficulté pour la force de vente consiste à
  expliciter simplement ce que permet de faire l'API à un public non
  technophile. Un outil de visualisation apporterait un atout fort
  pour accompagner leur discours et conquérir de nouveaux clients.
\end{frame}

\begin{frame}
  \frametitle{Avantage concurrentiel}

  Proposer des services nous permettant de nous différencier de la
  concurrence, aujourd'hui aucun concurrent direct ne propose des
  outils d'aide à la compréhension ou au prototypage aussi abouti que
  le projet APiHM.
\end{frame}

\section{Business model}

\begin{frame}
  \frametitle{Business model}

  La solution APiHM s'inscrit comme un service supplémentaire à
  intégrer dans l'offre de service navitia.io. Il vient alimenter le
  business model par la mise à disposition ou la vente de services
  additionnels.

  Freemium, paliers de consommation, accès aux services.

  Gains internes.
\end{frame}

\section{Organisation}

\begin{frame}
  \frametitle{Roadmap}

  Mixer UX et Agile:

  {\centering\includegraphics[width=0.7\linewidth]{images/roadmap}

  }

  Besoins matériels: des ordinateurs, connexion internet, coach
  développement web front.
\end{frame}

\begin{frame}
  \frametitle{L'après SWAT}

  \begin{itemize}
  \item Intégration d'APiHM dans le projet navitia.io
  \item APiHM en adéquation avec l'objectif 3 de Kisio Digital:
    navitia.io est la plateforme de mobilités de référence en France.
  \end{itemize}
\end{frame}

\section{Conclusion}

\begin{frame}
  \frametitle{Conclusion}

  APiHM:
  \begin{itemize}
  \item besoin avéré;
  \item réaliste;
  \item utile;
  \item par une équipe pluridisciplinaire;
  \item intégré rapidement à Kisio Digital.
  \end{itemize}
\end{frame}

\begin{frame}
  \titlepage
\end{frame}

\end{document}
